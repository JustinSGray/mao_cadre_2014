
One key technology that enables the flexible solution of an MDAO problem is the representation
of its dataflow as a dependency graph. A dependency graph is a directed network graph whose
edges represent the dependencies between its nodes. In a framework, an input needed by one
discipline must be provided by the output of another. This is a dependency, and is represented 
in a dependency graph containing nodes for the output and input, and an directional edge
between them. It is clear that the upstream discipline must provide the output before the 
downstream one can receive it, and this directional information is contained in the dependency
graph.

The main advantage of tight integration with a dependency graph is the availability of efficient
algorithms for performing a number of functions that are useful for solving an MDAO problem.

  1. Determination of component execution order.
  2. Identification of cycles.
  3. Identification of parallel structures.
  4. Determination of driver subgraphs.
  
The OpenMDAO framework uses a dependency graph to determine component execution order and to
drive the process of invalidation, which finds the minimum set of components that needs to
be re-executed when a set of inputs change. (ref: last OpenMDAO paper.) A dependency graph
can also identify cycles in the graph, and hence a cycle in the dataflow that must be resolved
by a solver. Similarly, the graph can also be used to examine the potential for paralellism at
a component level.

The determination of driver subgraphs is a capability that will be investigated in this paper.
Consider an optimizer with n parameters, 1 objective, and m constraints. The relevant 
subgraph of this problem is the set of component disciplines and their variable connection that
lie within the subgraph betwen the parameters and the contraints and objective. This subgraph
contains the components that execute when the optimizer requests a function evaluation. It is
also useful for setting up and solving the coupled derivative system when the optimizer requests
a gradient evaluation. For calculation of the coupled derivative in adjoint mode, further
efficiency can be gained by using the subgraph that includes just the portion of the graph 
between a single constraint and the parameters. Similarly for forward mode, the subgraph would
include just the portion of the graph between a single parameter and the constraints/objective.

[Intro to OpenMDAO paragraph]

[Derivatives in OpenMDAO paragraph]
