\documentclass[]{aiaa-tc} % insert '[draft]' option to show overfull boxes

 \title{Application of Graph Based Problem Formulation to Gradient Based Optimization Of Very Large Design Spaces In OpenMDAO}
        
\author{
  Justin Gray,%
     \thanks{Aerospace Engineer, MDAO Branch, Mail Stop 5-11, AIAA Member}
  \ Kenneth T. Moore,%
     \thanks{Senior Systems Engineer, MDAO Branch, Mail Stop 500-105, AIAA Senior Member}  
   \ Tristan A. Hearn,%
     \thanks{Aerospace Engineer, MDAO Branch, Mail Stop 5-10, AIAA Member}\\
  {\normalsize\itshape
  NASA Glenn Research Center, Cleveland, OH}  \\
 }

\AIAAconference{Multidisciplinary Design Optimization Specialist Conference}
\AIAAcopyright{\AIAAcopyrightD{2012}}


% Define commands to assure consistent treatment throughout document
\newcommand{\eqnref}[1]{(\ref{#1})}
\newcommand{\class}[1]{\texttt{#1}}
\newcommand{\package}[1]{\texttt{#1}}
\newcommand{\file}[1]{\texttt{#1}}
\newcommand{\BibTeX}{\textsc{Bib}\TeX}

\setlength{\abovecaptionskip}{0pt}
\setlength{\belowcaptionskip}{0pt}

\usepackage{setspace}

\usepackage{graphicx}
\usepackage{wrapfig}
\usepackage{caption} 
\usepackage{amsmath}
\usepackage{lscape}
\usepackage{hyperref}
\usepackage{minted}
\usepackage{color}
\usepackage{appendix}
\usepackage[section]{placeins}


\captionsetup[figure]{margin=5pt,font=small,labelfont=bf,textfont=bf,justification=justified,}
%\captionsetup[wrapfigure]{margin=5pt,font=small,labelfont=bf,justification=justified,singlelinecheck=off}
\captionsetup[table]{margin=5pt,font=small,labelfont=bf,textfont=bf,justification=justified,position=top}

\bibliographystyle{aiaa}

\usepackage{lettrine}
\usepackage{verbatim}

\begin{document}

  \maketitle
   
  \begin{abstract}

  \end{abstract}

  \section{Introduction}

    Many of today's the most intersting design problems involve very large design spaces with 100's or 1000's of 
    design variables. Large design spaces are often approached through the use of gradient based optimziation 
    analytic derivatives to achieve highly scalable solution stratgies. For instance, adjoint based gradient 
    methods have allowed CFD based shape optimization to tackle problems with 100's of design varibles. [Cite Juan's Groups
    work here]. Coupled Aero-structural optimziation is another area where gradient based optimization methods have 
    been employed. [Cite Martins groups work here]. Although these problems have large design spaces, 
    they include only a few disciplines (i.e. Geometry, Aerodynamics, Structures). The relative simplicity of 
    the problem formulation make it feasible to use custom implementations taylored to a specific problem. These specific 
    implemnetations make it difficult to modify the problem formulation without significant effort. For problems 
    with more disciplines customs solutions become less feasible, and a more general means of setting up 
    the design problem is required. 

    In this work we demonstrate how OpenMDAO, an open-source MDAO framework, provides a more general 
    solution to constructing and solving complex problems with large design spaces. Use of OpenMDAO 
    provides benefits in a number of important respects: 

    \begin{enumerate}
      \item Combining analytic derivatives with finite difference in a mixed derivatives environment
      \item Automatic and efficient solving for the coupled derivatives at the system level 
      \item Automatic assembly of the full gradient from the partial derivatives of each discipline
      \item Flexbility via separation of problem formulation from solution strategy 
      \item Simple implementation for multi-point design problems
    \end{enumerate}

    The approach used in OpenMDAO is maintain a central, monolithic, data connectivity graph between all 
    variables and components in the model. The graph utilizes the structure proposed by Pate et. al \cite{graph_problem2013} 
    and represents the complete problem formulation as defined by the user. The graph is then used in order to 
    facilitate all of the five features described above. One of the key aspects 

  \section{Problem Formulation Graph}

    One key technology that enables the flexible solution of an MDAO problem is the representation
    of its dataflow as a dependency graph. A dependency graph is a directed network graph whose
    edges represent the dependencies between its nodes. In a framework, an input needed by one
    discipline must be provided by the output of another. This is a dependency, and is represented 
    in a dependency graph containing nodes for the output and input, and an directional edge
    between them. It is clear that the upstream discipline must provide the output before the 
    downstream one can receive it, and this directional information is contained in the dependency
    graph.

    The main advantage of tight integration with a dependency graph is the availability of efficient
    algorithms for performing a number of functions that are useful for solving an MDAO problem.

      1. Determination of component execution order.
      2. Identification of cycles.
      3. Identification of parallel structures.
      4. Determination of driver subgraphs.
      
    The OpenMDAO framework uses a dependency graph to determine component execution order and to
    drive the process of invalidation, which finds the minimum set of components that needs to
    be re-executed when a set of inputs change. (ref: last OpenMDAO paper.) A dependency graph
    can also identify cycles in the graph, and hence a cycle in the dataflow that must be resolved
    by a solver. Similarly, the graph can also be used to examine the potential for paralellism at
    a component level.

    The determination of driver subgraphs is a capability that will be investigated in this paper.
    Consider an optimizer with n parameters, 1 objective, and m constraints. The relevant 
    subgraph of this problem is the set of component disciplines and their variable connection that
    lie within the subgraph betwen the parameters and the contraints and objective. This subgraph
    contains the components that execute when the optimizer requests a function evaluation. It is
    also useful for setting up and solving the coupled derivative system when the optimizer requests
    a gradient evaluation. For calculation of the coupled derivative in adjoint mode, further
    efficiency can be gained by using the subgraph that includes just the portion of the graph 
    between a single constraint and the parameters. Similarly for forward mode, the subgraph would
    include just the portion of the graph between a single parameter and the constraints/objective.

    [Intro to OpenMDAO paragraph]

    [Derivatives in OpenMDAO paragraph]
 
  \bibliography{references}

\end{document}
